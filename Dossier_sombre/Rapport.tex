\documentclass{article}
\usepackage[utf8]{inputenc}
\usepackage{minted}
\usepackage{lmodern}
\title{Projet Algo}
\author{Corentin Cornou }
\date{Octobre 2022}

\begin{document}

\maketitle


\section{Introduction}
\subsection{Type my\_list}
\begin{minted}{Ocaml}
type 'a my_list =
| Nil
| Cons of 'a * 'a my_list
\end{minted}

The type my\textunderscore list.\\

For example : \mint{Ocaml}|Cons(1, Cons(2, Nil))| is equivalent to the list [1,2].

\bigskip

\subsection{Function string\_of\_list}
\begin{minted}{Ocaml}
let string_of_list str_fun l =
  let rec string_content = function
    | Nil  -> ""
    | Cons(x, Nil)  -> (str_fun x)
    | Cons(x,q) -> (str_fun x) ^ ", " ^ (string_content q)
  in "[" ^ (string_content l) ^ "]" ;;
\end{minted}

Create a string from the my\textunderscore list object.
For example : 
\mint{Ocaml}|string_of_list string_of_int (Cons(1, Cons(2, Nil)))|
returns "[1,2]"\\

\newpage
\subsection{Function hd}

\begin{minted}{Ocaml}
let hd l =
  match l with
  | Nil -> None
  | Cons (x,_) -> Some( x );;
\end{minted}

Returns the head of a my\_list element.\\

For example : \mint{Ocaml}|hd (Cons(1, Cons(2, Nil)))| returns $1$\\

\medskip


\subsection{Function tl}


\begin{minted}{Ocaml}
let tl l =
  match l with
  | Nil -> None
  | Cons ( _ , Nil ) -> None
  | Cons ( _ , q ) -> Some(q);;

\end{minted}

Returns the tail of a my\_list element.\\

For example : \mint{Ocaml}| hd (Cons(1, Cons(2, Cons(3,Nil))))| returns \mint{Ocaml}|Cons(2, Cons(3,Nil))|
\medskip

\newpage

\subsection{Function length}

\begin{minted}{Ocaml}
let rec length l =
  match l with
  | Nil -> 0
  | Cons( _ , q)  -> 1 + length q;;
\end{minted}
Returns the length (i.e. number of elements) of a my\_list element.\\

For example : \mint{Ocaml}|length (Cons(1, Cons(2, Cons(3,Nil))))| returns 3\\

\subsection{Function map}

\begin{minted}{Ocaml}
let rec map f l  =
  match l with
  | Nil -> Nil
  | Cons ( x , q ) -> Cons ( f x , map f q );;
\end{minted}

\mint{Ocaml}|map f l| returns the my\_list element made by applying the \textbf{f} function to all elements in \textbf{l}.

For example : \mint{Ocaml}|map (fun x -> x*x) (Cons(1, Cons(2, Cons(3,Nil))))| returns  \mint{Ocaml}|Cons(1, Cons(4, Cons(9,Nil)))|
%\input{code.tex}

\end{document}
